% TEX program=xelatex
\documentclass[10pt,twocolumn,letterpaper]{article}

\usepackage{cvpr}
\usepackage{times}
\usepackage{epsfig}
\usepackage{graphicx}
\usepackage{amsmath}
\usepackage{amssymb}
\usepackage{enumitem}

% Include other packages here, before hyperref.

% If you comment hyperref and then uncomment it, you should delete
% egpaper.aux before re-running latex.  (Or just hit 'q' on the first latex
% run, let it finish, and you should be clear).
\usepackage[breaklinks=true,bookmarks=false]{hyperref}
\cvprfinalcopy % *** Uncomment this line for the final submission

\def\httilde{\mbox{\tt\raisebox{-.5ex}{\symbol{126}}}}

% Pages are numbered in submission mode, and unnumbered in camera-ready
%\ifcvprfinal\pagestyle{empty}\fi
\setcounter{page}{1}
\begin{document}

%%%%%%%%% TITLE
\title{Generating Anime Avatars Using GANs \\ \large Final Report for Deep Learning Course}

\author{Wang Ao\\
Tsinghua University\\
2017010395\\
{\tt\small wa17@mails.tsinghua.edu.cn}
% For a paper whose authors are all at the same institution,
% omit the following lines up until the closing ``}''.
% Additional authors and addresses can be added with ``\and'',
% just like the second author.
% To save space, use either the email address or home page, not both
\and
Sun Ziping\\
Tsinghua University\\
2015013249\\
{\tt\small sunziping2016@yeah.net}
\and
Cui Yanfei\\
Tsinghua University\\
2017012326\\
{\tt\small 929881841@qq.com}
}

\maketitle
%\thispagestyle{empty}

%%%%%%%%% ABSTRACT
\begin{abstract}
% Goal, Approach, Novel Contribution, Conclusion
In the past few years, computer vision has achieved rapid development, and the
Generative Adverarial Network~\cite{GAN} has attracted wide attention since
proposed in 2014. In this paper, we aimed to generate from noise and to transfer
from existing pictures to anime avatars, which is an interesting and practical
task. There have been many paper achieving this kind of task, such as
DCGAN~\cite{DCGAN}, CartoonGAN~\cite{CartoonGAN} and
CycleGAN~\cite{CycleGAN2017}. In this paper, we reimplement these three models
and tune them to achieve better performance on this specific task. And then, we
give several novel loss functions to make models perform better, such as face
key point loss, greyscale loss and color loss. At last, we compare the different
results between these three models and come to the conclusion that DCGAN is
generally a good choice to generate images from noise, CycleGan performs the
best on image tansfering while CartoonGAN tends to preserve image details, which
results to a poorer performance.
\end{abstract}

%%%%%%%%% BODY TEXT
\section{Introduction}

% GAN, DCGAN, CycleGAN, Evalutaion method
Generally speaking, image generating and style transferring belongs to the
generative problem. And deep generative models have achieved impressive success
in this field. Among all deep generative models, the most outstanding ones, for
the time being, are Generative Adversarial Networks (GANs) and Variational
Autoencoders (VAEs). In our work, we focus on GANs.

Traditional GANs consist of two parts, the generator and the discriminator. The
generator tries to generate fake outputs similar to the real data, while the
discriminator tries to tell the real ones apart from the fake ones. When they
trained together, the discriminator drives the generator to generate realer
outputs, and meanwhile, the generated realer outputs also drive the
descriminator to jugde more correctly. And finally, in order to deceive the
discriminator, the generator can be well-trained for the generating task, that's
exactly what we want. It is also notable that these two parts may not
necessarily be trained exactly simultaneously. One part can be pretrained, or
they can be trained alternatively.

As mentioned above, there are two ways to generate anime avatars:

\begin{itemize}[noitemsep, topsep=0pt]
   \item generate: directly generate a anime avatar from random noise
   \item transfer: generate corresponding anime avatar from real human avatar
\end{itemize}

For the first approach, DCGAN is among one of the most state-of-the-art models.
It is a variation of GAN, in which transposed convolutions are added. It is
mainly aimed for image generating tasks. Transposed convolutions are the inverse
operation of normal convolutions, and they just broadcast input values through
the kernels, resulting a larger output shape, which is useful for up-sampling.

The character's head of the output picture is very clear, the lines are very coherent, and there
is no serious deformation. But it may be because the input is random noise and the data
set is not large enough, the homogeneity of the generated content is more serious, and the
characters always look like the same. At the same time, it is difficult for us to have a better
evaluation index to judge the effectiveness of the model. In addition, the training of GAN
is very unstable, which leads to the need for a large amount of hyperparameter tuning work.
{\color{red}Continue adding more information about direct generating...}

As for the second approach, CartoonGAN~\cite{CartoonGAN} is able to successfully convert landscape photos
into anime style. When we apply CartoonGAN~\cite{CartoonGAN} to the task of character head style transfer,
we need a new dataset, a new feature extraction module, and an adaptation to the face.
In the feature extraction module, we use the classic VGG19~\cite{VGG} as the feature extraction
module of the picture to extract the animation elements. For the human face adaptation,
we added a landmark loss function of the human face to mark the key points of the human face and
help the discriminator to better identify whether the generated image is a human face. In addition,
we have also added style loss function, image content loss function,
grayscale loss function and color loss function, making it less difficult to
judge the quality of image generation.

%------------------------------------------------------------------------
\section{Related Work}

\subsection{Non-photorealistic rendering (NPR)}
In order to mimic specific artistic styles (including animation~\cite{rosin2012image}), many automatic and
semi-automatic NPR algorithms have been developed. Most of the works are animated by
using a simple shadow rendering method~\cite{saito1990comprehensible}. One technique, called  \textsl{cel shading}, is widely
used in the creation of games, animations, and movies, saving artists a lot of time~\cite{luque2012cel}.

To mimic the cartoon style, people have developed various methods to create
images with flat shadows. These methods either use image filtering~\cite{winnemoller2006real} or use specific
transformation formulas~\cite{xu2011image} in optimization problems. However, it is difficult to
capture rich artistic styles using only simple mathematical formulas. In
particular, filtering or optimizing the entire image does not create the
high-level abstractions that artists typically require. For portraits, people
also have special algorithms~\cite{yang2010semantics, rosin2015non}, in which semantic segmentation can be automatically
derived by detecting facial components. However, these methods cannot handle general images.

\subsection{Convolutional neural networks}
With the great success of convolutional neural networks~\cite{krizhevsky2012imagenet, lawrence1997face} in the field of
computer vision, people are looking forward to their performance in the
field of image style transfer and image generation. Compared with the
traditional complex NPR algorithm~\cite{saito1990comprehensible,luque2012cel}, CNN is indeed more convenient and more
applicable. For example, in the task of image style transfer, the VGG
network~\cite{VGG} has a good ability to extract picture features.

For the style and content of the image, Gatys et al.~\cite{NST} first proposed a
CNN-based neural style transfer (NST) method that can transfer the style
of a picture from one picture to another. They use a feature map of a
pre-trained VGG network to extract picture content and optimize the
resulting image, so that it can match the corresponding texture information
described by the global Gram matrix~\cite{gatys2015texture} while retaining the original content
of the image. However, such operations caused a serious loss of the edge
information and shadow information of the picture.

\subsection{Image synthesis with GANs}
The other genre using GANs~\cite{GAN} has achieved great imporvement. It has achieved
the state of the art results in the fields of text-to-image translation~\cite{reed2016generative}, image
inpainting~\cite{yeh2016semantic}, and image super-resolution~\cite{ledig2017photo}, etc. The key idea of the GAN model
is to train two networks (generator and discriminator). Iteratively,
the adversarial loss provided by the discriminator transforms the
generated image into a target manifold~\cite{yeh2016semantic}. However, GANs are very unstable to train and
often make the generator produce meaningless output.

Some literatures~\cite{dumoulin2016adversarially,isola2017image,karacan2016learning} provided solutions using GANs for pixel-level image
synthesis problems. However, these methods require paired data sets
during the training process, but such high-quality data sets are
difficult to obtain and therefore cannot be used for our training.

On the one hand, CartoonGAN effectively solved the above problems by using a GAN model
to learn the mapping between photos using unpaired training data and
cartoon manifolds. They formulate the process of learning to transform realworld
photos into cartoon images as a mapping function which maps the photo
manifold to the cartoon manifold~\cite{CartoonGAN}. Additionally, they proposed
edge-promoting adversarial loss in order to generate sharper edges just like anime works.
However, their work places high demands on the quality of the training set, the input
figure must have a high color saturation to get anime-like outputs. At the same time,
it doesn't provide a good method to guarantee that the image is a portrait.

On the other hand, DCGAN~\cite{DCGAN} made some great improvements on the basis of traditional GAN.
It set a series of restrictions for the network topology of CNN to make it stable training, and
used the obtained feature representations for image classification so that it get better results
to verify the expression ability of the generated image feature representations.
{\color{red}Continue adding more information about DCGAN's advantages and limitation...}


%------------------------------------------------------------------------
\section{Approach}
\subsection{Generate from Noise Using DCGAN}
A DCGAN model


\subsection{Transfer from Real Avatar Using CycleGAN}

%------------------------------------------------------------------------
\section{Experiment}

%------------------------------------------------------------------------
\section{Conclusion}

{\small
\bibliographystyle{ieee}
\bibliography{egbib}
}

\end{document}
